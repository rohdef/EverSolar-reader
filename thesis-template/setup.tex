Setup which is used in this research paper is very simple. There are actually two setups, because there were a need to record cloud coverage data from somewhere, and at the same time there were a need to record solar panel data.\\ 
First setup which collects solar panel data is placed in a town named Mønsted which can be seen on a picture below. The setup consists of  PV, inverter, and raspberry pi, which collects data into a Sql database. \\
The second setup which records cloud coverage data is placed inside Aarhus university, where a stationary computer is running and gathering data from OpenWeather.com api, and stores the data into separate sql database. Data which is gathered from OpenWeather api are coming from two different towns Stoholm and Karup where both towns are closest to first setups placement, and there are no way to force OpenWeather api to return data from the same town always.\\
At last data from both setups are read into one main DB where data gets treated. 

\begin{figure}[h]
  \centering
 % \includegraphics{MapsPicture.png}
 \includegraphics{dummy.jpg}
  \caption{3 towns where data is recorded}
      \label{fig:MapsPicture}
\end{figure}