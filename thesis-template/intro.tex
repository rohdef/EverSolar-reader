There are several types of clean power sources in the world, solar energy, wind power, and water energy.
 This research paper takes photovoltaic modules (PV) as point of interest. PV commonly known as solar panels, works by
having two silicon based semiconductors, namely a n-type and a p-type, which are great conductors in some circumstances and not so good in others. Semiconductors are surrounded by front and back contact, which allows for better energy flow and when exposed to photons which is used to
create electricity \citep{photovoltaic}. Finally to protect solar cell from damage a thick glass is placed on top. Since PV generated energy cant be use directly because the energy generated by a solar cell is DC energy, it has to go through a inverter which converts DC energy to AC energy.

It has already been found that weather data to is relevant for any clean energy generating systems. For PV
generating systems, when maximum power point tracking is used to
maximize the power output, mainly temperature and radiation \citep{mppt2004}.

\todo{BAsed on this we want to try to correlate power generated with cloud conditions}


%%% Local Variables:
%%% mode: latex
%%% TeX-command-extra-options: "-shell-escape"
%%% TeX-master: "thesis"
%%% End:
