Determining weather data from various sensors and projecting expected
sensor data from weather sources is subject to a lot of interest.
Especially in the context of internet of things, where weather data is
relevant to estimate the throughput from natural energy sources such
as solar panels, also known as photovoltaic (PV) systems and wind
mills.  Estimating weather data from sensors may currently have a
limited use in practice, but could - provided sufficient precision -
be used to assist weather stations or be part of the Internet of
Things (IoT) vision with full home automation.

Recently a lot of people have invested in putting PV systems on their
roofs, to save money in the long run.  From the IoT perspective this
provides a potential sensor, as the inverter modules has
interfaces for data logging.  Since a system is mounted outside and
reacts to sunlight this seems like a potential sensor for weather
information.

PV systems work by having two silicon parts (N-type and P-type) that,
when exposed to photons, have an electron reaction that can be used to
produce electricity.  The fact that photons are needed to produce the
power makes it likely that there is a connection between things that
affect the amount of light reaching the panels and their effect, such
as clouds.  When the silicon is doped to produce the N-type and P-type
parts it is doped with materials that react with the longer
wavelengths of light, i.e. closer to the infrared wavelengths, to
minimize the impact from atmospheric disturbances and clouds
\citep{photovoltaic}.  Since it's only minimizing it indicates that
there is at least some effect.

When the solar panels generate electricity it is a direct current,
which is sent to an inverter, which is used to transform the power current to
an alternating current, and controls if it should transfer the power back
to the energy grid and most interestingly provides a serial interface
for data logging equipment.

Since research has already shown that PV systems are affected by
weather conditions such as temperature and cloud coverage
\citep{mppt2004,southafrica,cloudTrack}, it seems reasonable that it's
possible to track cloud data from the current effect of the panels,
thus enabling estimates of the cloud coverage from the effect or
similarly estimate the effect based on expected cloud coverage.

%%% Local Variables:
%%% mode: latex
%%% TeX-command-extra-options: "-shell-escape"
%%% TeX-master: "thesis"
%%% End:
