That a single PV system is very sensitive to even small changes can
easily be seen following the green curve on figure \ref{fig:noise}.
It's apparent from the amount of fluctuations that even small things
can influence the effect.  Both \citep{cloudTrack} and
\citep{southafrica} has data confirming that small scale PV systems
are easily affected by things such as clouds, they still look at large
systems in comparison with a personal system.  Thus it is likely that
even a flock of birds can influence it.

A larger area of PV systems and bigger PV systems can smooth out noise
in the data \citep{southafrica,cloudTrack}.  The effect of larger
systems could be interesting if they can be accessed, for the approach
of this project it would be an interesting addition to get access to
more systems.  The approach to have a larger area of PV systems is
used is used to try and even out all noise caused from clouds
\citep{southafrica}, where the intend here would be to even out some,
such that conclusions about correlation about the weather and effect
can be made.  The fact that the weather impact can be smoothed a lot,
when it's done over areas of 250x250 and 500x500 km squares, but
weather data is still quite apparent on 50x50 km squares
\citep{southafrica}, indicates both that a larger area than a single
PV system will smooth the data, but without loosing the stronger
connection with weather data as long as the area is relatively small.

%%% Local Variables:
%%% mode: latex
%%% TeX-command-extra-options: "-shell-escape"
%%% TeX-master: "thesis"
%%% End:
