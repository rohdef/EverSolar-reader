Given that PV systems react to photons and thus are affected by clouds
and atmospheric disturbances indicates that we should be able to make
a connection between clouds and the effect of the panels
\citep{photovoltaic}.  This claim is further backed by research being
done to even out the noise from weather conditions, most notably
clouds \citep{southafrica, cloudTrack}.  However the fact that the
data analysis contrasts this leaves out three possible conclusions: 1)
there actually is no such connection, 2) clouds do impact but too many
other factors impact it to make any reliable information from clouds
alone, and 3) there are errors in the way the data is collected.  As
option seems very unlikely, both from the science behind PV systems
and that research groups seem to be able to prove this
\citep{southafrica, cloudTrack, photovoltaic}, it seems appropriate to
take a look at option 2 and 3.

Considering that there may be too many data influencing the
measurements, as to cloud coverage by itself being a too small factor
to actually show the intended connection between cloud coverage and
the effect of the system, there is a few options to remedy this.  From
the fact that temperature impacts the amount of power a PV system
produces \citep{mppt2004} taking further account of these data may
show an improvement.  This suspicion however is weakened from the mean
and deviation plots in figure \ref{fig:stat0610}, \ref{fig:stat1014}
and \ref{fig:stat1418}, as the temperature should even out when using
measurements at the same times of the day.  The data on that account
though is in no way conclusive, and a further investigation into it
would require a lot more data than available in this study.

That a single PV system is very sensitive to even small changes can
easily be seen following the green curve on figure \ref{fig:noise}.
It's apparent from the amount of fluctuations that even small things
can influence the effect.  Attempts to even out the effect from
weather conditions, has successfully shown that both larger PV systems
and larger area coverage can be used smooth the data from weather
conditions \citep{cloudTrack,southafrica}.  This heavily indicates
that using more PV systems could be useful.  Since the target here
isn't to completely even the data, the balance between still having
data and reducing the noise gathered.

The approach to have a larger area of PV systems is used is used to
try and even out all noise caused from clouds.  Using large areas with
multiple PV systems shows that areas of 250x250 and 500x500 km squares
even out very large amounts of the data, but weather data is still
quite apparent on 50x50 km squares \citep{southafrica}.  This
indicates that using a larger area with more PV systems is a viable
approach.  Doing this in a small city such as the one where the
experiment is done would be fairly easy provided that people would
allow collection of their data.

Considering the claim that 1 MW solar panels responds rapidly to
changes in available sunlight \citep{cloudTrack} it is quite likely
that one of the data in the experiment is quite simply too sensitive.
The usage of larger PV systems isn't a likely approach if using
peoples personal systems on their roofs, since a personal system in
Denmark is allowed to be max 3 kW, which is a fair deal lower than the
1 MW system.  This however indicates that even very small disturbances
will have an effect of the solar panels and thus simple renders faulty
data.

Another possible cause for faulty data is the lack of precision of the
weather data.  The data given for free from OpenWeatherMap.org is
guaranteed to be updated a the latest every 2 hours.  Also the weather
station doing the measurement isn't located exactly where the PV
system is.  So it is quite possible that the ground truth data simply
isn't good enough.

On a final note it would also be an improvement to have data over a
longer period of time, as this would assist in either making it
certain that the noise is either consistent or that it would over time
start to show stronger tendencies towards a connection.

There is plenty of room for improving the data gathered, both by
trying to reduce the chance for potential direct errors but also to
make other effects more apparent.  From the data here though it's too
early to conclude the lack of a connection between clouds an the
effect from solar panels.

Another point of improvements is to actually find out sun's role in energy production of PV, because every PV have an optimal sun angle where they receive optimal amount of sun light. Since optimal sun angle are known, suns placement on the sky at gathered data timestamps can be found, to calculate if there are actual a link between suns angle and PV power generation. 

%%% Local Variables:
%%% mode: latex
%%% TeX-command-extra-options: "-shell-escape"
%%% TeX-master: "thesis"
%%% End:
