%\subsection{Weather sensing}
The connection between PVs and the effect that weather, most notably
clouds, have on them have already been explored, but from the
perspective of minimizing the impact from weather data.  One of the
researches explored the impact from the size of the solar panels
\citep{cloudTrack} whereas the other focused on the impact from
gathering data from multiple solar power plants spread over large
areas \citep{southafrica}.  Not surprisingly the studies show that
bother larger panels and combining the data from spread panels even
out effects, such as clouds.

Maximum power point tracking is a technique that the inverters of PV
systems use to maximize power output.  Research has been done to
optimize the algorithms based on weather data, since the heat also has
an effect on the power produced \citep{mppt2004}.  The research shows
that indeed the temperature can be used to optimize the algorithms.

From the IoT perspective it's also interesting what other weather data
can be gathered from less conventional sources, this could potentially
be part of a more overall solution, utilizing the data for various
kinds of automation or feeding it back to some services, for special
purposes or for the general good.  It is for instance possible to
imagine (a bit visionary) the power plans adjusting to more accurate
live data based on projections from a large amount of data gathered
like this.

OpenSignal have created the app WeatherSignal that utilized phone
battery temperatures \citep{temperatures2013} to deduce information
about the air temperature in heavily populated areas.  They proved a
strong relation between the temperature of a the phone batteries and
the outside temperatures.  The authors believe that the data can be
further improved, if the data can be calibrated according to phone
models and if the phone is inside or outside, as heating and air
conditions bias the data.

A correlation between the radio signals used for phones and rain
measurements has also been proved \citep{rainfall2007}.  The radio
signals can be used to measure the amount of rain in between two
points.  The measurements is fairly accurate, and it's assumed that
they can be further improved if improved for the signal loss due to
wet antennas.

%%% Local Variables:
%%% mode: latex
%%% TeX-command-extra-options: "-shell-escape"
%%% TeX-master: "thesis"
%%% End:
