%\subsection{Weather sensing}
There has already been some research in measuring weather data by
utilizing some of the sensor capabilities that arises from the
increasing amount of technology in use.

OpenSignal have created the app WeatherSignal that utilized phone
battery temperatures \citep{temperatures2013} to deduce information
about the air temperature in heavily populated areas.  They proved a
strong relation between the temperature of a the phone batteries and
the outside temperatures.  The authors believe that the data can be
further improved, if the data can be calibrated according to phone
models and if the phone is inside or outside, as heating and air
conditions bias the data.

A correlation between the radio signals used for phones and rain
measurements has also been proved \citep{rainfall2007}.  The radio
signals can be used to measure the amount of rain in between two
points.  The measurements is fairly accurate, and it's assumed that
they can be further improved if improved for the signal loss due to
wet antennas.

It is believe that the combination of various weather sensing
techniques such as \cite{rainfall2007}, \cite{temperatures2013} and
measurement of clouds by PV systems, such as proposed in this article,
is likely to have some useful potential.  For instance the maximum
power point tracking units of PV systems can be improved with weather
data \citep{mppt2004}.

%%% Local Variables:
%%% mode: latex
%%% TeX-command-extra-options: "-shell-escape"
%%% TeX-master: "thesis"
%%% End:
