From the fact that PV systems is sensitive to the amount of photons
reaching them and that research has shown that larger scale PV systems
and areas can be used to smooth out weather data the expected results
show a clear connection \citep{southafrica, cloudTrack, photovoltaic}.
The approach will be to first off confirm that it's likely to be true
and then refining the data for the purpose to confirm or refute this
hypothesis.

By plotting the raw data
The apparent view of things the data also seem to have a correlation
as can be seen on figure \ref{fig:cloudsAndPower}.  It seems likely
from the graph that there is some correlation, between the clouds and
the PV system, as there's a tendency for the effect to drop when
there's more clouds.

\begin{figure}
  \centering
  \includegraphics{dummy.jpg}
  \caption{Noisy unfiltered data}
  \label{fig:noise}
\end{figure}

\begin{figure}
  \centering
  \includegraphics{dummy.jpg}
  \caption{Cloud cover and averages of power generated (10 minute
    intervals)}
  \label{fig:cloudsAndPower}
\end{figure}

However further analysis puts some doubt to that.  The blue line in
figure \ref{fig:cloudsAndPower} shows the optimal effect measured at a
given point of time in the day.  To test that hypothesis a scatter
plot comparing how far from the optimal the power generation is
compared to the cloud coverage.

\begin{figure}[h]
  \centering
  \includegraphics{dummy.jpg}
  \caption{Scatter plot from comparing effect, cloud coverage and
    optimal effect}
  \label{fig:fractiles}
\end{figure}

\todo{Boxplot comments, more analysis}

That a single PV system is very sensitive to even small changes can
easily be seen following the green curve on figure \ref{fig:noise}.
It's apparent from the amount of fluctuations that even small things
can influence the effect.  Both \citep{cloudTrack} and
\citep{southafrica} has data confirming that small scale PV systems
are easily affected by things such as clouds, they still look at large
systems in comparison with a personal system.  Thus it is likely that
even a flock of birds can influence it.

A larger area of PV systems and bigger PV systems can smooth out noise
in the data \citep{southafrica,cloudTrack}.  The effect of larger
systems could be interesting if they can be accessed, for the approach
of this project it would be an interesting addition to get access to
more systems.  The approach to have a larger area of PV systems is
used is used to try and even out all noise caused from clouds
\citep{southafrica}, where the intend here would be to even out some,
such that conclusions about correlation about the weather and effect
can be made.  The fact that the weather impact can be smoothed a lot,
when it's done over areas of 250x250 and 500x500 km squares, but
weather data is still quite apparent on 50x50 km squares
\citep{southafrica}, indicates both that a larger area than a single
PV system will smooth the data, but without loosing the stronger
connection with weather data as long as the area is relatively small.

%%% Local Variables:
%%% mode: latex
%%% TeX-command-extra-options: "-shell-escape"
%%% TeX-master: "thesis"
%%% End:
