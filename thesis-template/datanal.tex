The expected data is to see a clear connection between the output of a
PV and clouds\citep{southafrica}.  The aparant view of things the data
also seem to have a correlation as can be seen on figure
\ref{fig:cloudsAndPower}.  It seems likely from the graph that there
is some correlation, between the clouds and the PV system, as there's
a tendency for the effect to drop when there's more clouds.

\begin{figure}[h]
  \centering
  \includegraphics{dummy.jpg}
  \caption{Cloud cover and averages of power generated (10 minute
    intervals)}
  \label{fig:cloudsAndPower}
\end{figure}

However further analysis puts some doubt to that.  The blue line in
figure \ref{fig:cloudsAndPower} shows the optimal effect measured at a
given point of time in the day.  To test that hypothesis a scatter
plot comparing how far from the optimal the power generation is
compared to the cloud coverage.

\begin{figure}[h]
  \centering
  \includegraphics{dummy.jpg}
  \caption{Scatter plot from comparing effect, cloud coverage and
    optimal effect}
  \label{fig:fractiles}
\end{figure}

\todo{Boxplot comments, more analysis}

Having a larger area of PV systems can smooth out noise in the data
\citep{cloudsAndPower} and it is thus likely that having an area with
more panels would even out the data in meaningful way.  Since we're
still interested in the cloud coverage, it would still have to be
local areas, such as a small village or similar.

%%% Local Variables:
%%% mode: latex
%%% TeX-command-extra-options: "-shell-escape"
%%% TeX-master: "thesis"
%%% End:
