From the fact that PV systems is sensitive to the amount of photons
reaching them and that research has shown that larger scale PV systems
and areas can be used to smooth out weather data the expected results
show a clear connection \citep{southafrica, cloudTrack, photovoltaic}.
The approach will be to first off confirm that it's likely to be true
and then refining the data for the purpose to confirm or refute this
hypothesis.

By plotting the raw data in figure \ref{fig:noise} it can be seen that
the cloud effect data from the PV system is indeed very noisy.  The
fluctuations are so extreme that in themselves they only tell us that
the solar panels are affected a lot by small things.  The cloud data
is less noisy, since OpenWeatherMap provides coarser data, and even if
they didn't it's fairly safe to assume that cloud coverage (at least
under normal circumstances) won't change radically in short intervals.

\begin{figure}
  \centering
  \includegraphics{dummy.jpg}
  \caption{The raw cloud and effect data, as can be seen the effect
    contains a huge amount of noise.}
  \label{fig:noise}
\end{figure}

To make the data correlate better and to even out the noise on the
effects, the data is averaged over time intervals.  This evens out a
lot of the small fluctuations, we consider it likely from the level of
sensitivity that it's things like small clouds, flocks of birds and
all the other small things that temporarily block sunlight that are
the causes.

This results in figure \ref{fig:cloudsAndPower}, where the maximal
effect from this time of the day during the period of data collection
is added.  There seems to be a hint of a correlation, as for most of
the data points with cloudy weather the effect seems to be further
from the optimal and with less cloudy closer to the optimal, but it
seems only to be a hint, since the data is still rather noisy.

\begin{figure}
  \centering
  \includegraphics{dummy.jpg}
  \caption{Cloud cover and averages of power generated (10 minute
    intervals)}
  \label{fig:cloudsAndPower}
\end{figure}

A lot of experiments were done to find the a way to make the data more
useful, as expected the intervals used doesn't make much of a
difference, as long as they are big enough to even out the worst
fluctuations (around 10 min) and small enough to stay without each
cloud measurement (<= 30 mins).  The lowest 5\% of the data has been
capped, since the fluctuations close to this margin is very noisy,
probably due to the low margins, experiments has been done with other
margins, but at 10\% the data doesn't change much and at higher rates,
it seems that too many correct data get filtered too.

As part of the refinement process a scatter plot was added, to make it
clearer if there's any relation.  The plot shows a hint of a relation,
but nothing more, it's way too vague to conclude anything.  In the
scatter plot on figure \ref{fig:scatter30} that uses 30 min intervals
the data is extremely noisy, whereas on scatter plot for 2 hour
intervals on figure \ref{fig:scatter120} seems to have a clearer
connection, but may have too few data points for reliable conclusions.

\begin{figure}
  \centering
  \includegraphics{dummy.jpg}
  \caption{Scatter plot from comparing effect, cloud coverage and
    optimal effect, using averages over 30 mins}
  \label{fig:scatter30}
\end{figure}

\begin{figure}
  \centering
  \includegraphics{dummy.jpg}
  \caption{Scatter plot from comparing effect, cloud coverage and
    optimal effect, using averages over 2 hours}
  \label{fig:scatter120}
\end{figure}

\todo{Boxplot comments, more analysis}

That a single PV system is very sensitive to even small changes can
easily be seen following the green curve on figure \ref{fig:noise}.
It's apparent from the amount of fluctuations that even small things
can influence the effect.  Both \citep{cloudTrack} and
\citep{southafrica} has data confirming that small scale PV systems
are easily affected by things such as clouds, they still look at large
systems in comparison with a personal system.  Thus it is likely that
even a flock of birds can influence it.

A larger area of PV systems and bigger PV systems can smooth out noise
in the data \citep{southafrica,cloudTrack}.  The effect of larger
systems could be interesting if they can be accessed, for the approach
of this project it would be an interesting addition to get access to
more systems.  The approach to have a larger area of PV systems is
used is used to try and even out all noise caused from clouds
\citep{southafrica}, where the intend here would be to even out some,
such that conclusions about correlation about the weather and effect
can be made.  The fact that the weather impact can be smoothed a lot,
when it's done over areas of 250x250 and 500x500 km squares, but
weather data is still quite apparent on 50x50 km squares
\citep{southafrica}, indicates both that a larger area than a single
PV system will smooth the data, but without loosing the stronger
connection with weather data as long as the area is relatively small.

%%% Local Variables:
%%% mode: latex
%%% TeX-command-extra-options: "-shell-escape"
%%% TeX-master: "thesis"
%%% End:
