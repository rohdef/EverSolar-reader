From the fact that PV systems are sensitive to the amount of photons
reaching them and that research has shown that larger scale PV systems
and areas can be used to smooth out weather data the expected results
show a clear connection \citep{southafrica, cloudTrack, photovoltaic}.
The approach will be to firstly confirm that it's likely to be true
and then refine the data for the purpose to confirm or refute this
hypothesis.

By plotting the raw data in figure \ref{fig:noise} it can be seen that
the cloud effect data from the PV system is indeed very noisy.  The
fluctuations are so extreme that they only tell us that the solar
panels are affected a lot by small things.  The cloud data is less
noisy, since OpenWeatherMap provides coarser data, and even if they
didn't it's fairly safe to assume that cloud coverage (at least under
normal circumstances) won't change radically in short intervals.

\begin{figure}
  \centering
  \tiny
  \includesvg[width=\textwidth]{noise}
  \caption{The raw cloud and effect data, as can be seen the figure
    contains a huge amount of noise.}
  \label{fig:noise}
\end{figure}

To make the data correlate better and to even out the noise on the
effects, the data is averaged over time intervals.  This evens out a
lot of the small fluctuations, we consider it likely from the level of
sensitivity that it's things like small clouds, flocks of birds and
all the other small things that temporarily block sunlight that are
the causes of the noise.

This results in figure \ref{fig:cloudsAndPower}, where the maximal
effect from this time of the day during the period of data collection
is added.  There seems to be a hint of a correlation, as for most of
the data points with cloudy weather the effect seems to be further
from the optimal and with less cloudy closer to the optimal, but it
seems only to be a hint, since the data is still rather noisy.

\begin{figure}
  \centering
    \tiny
    \includesvg[width=\textwidth]{diagram}
  \caption{Cloud cover and averages of power generated (10 minute
    intervals)}
  \label{fig:cloudsAndPower}
\end{figure}

A lot of experiments were conducted to find a way to make the data
more useful.  As expected the intervals used do not make much of a
difference, as long as they are big enough to even out the worst
fluctuations (around 10 min) and small enough to stay without each
cloud measurement (<= 30 mins).  The lowest 5\% of the data has been
capped, since the fluctuations close to this margin is very noisy,
probably due to the low margins. Experiments have been conducted with
other margins, but at 10\% the data do not change much and at higher
rates, it seems that too many correct data get filtered out.

As part of the refinement process a scatter plot was added, to make it
clearer if there's any relation.  The plot shows a hint of a relation,
but nothing more, it's way too vague to conclude anything.  In the
scatter plot on figure \ref{fig:scatter30} that uses 30 minute
intervals the data is extremely noisy, whereas on scatter plot for 60
minute intervals on figure \ref{fig:scatter60} seems to have a clearer
relation between the cloud cover an effect, but it is still too noisy.
A trial was done with even coarser data, but it turned out to have so
few data points that is was virtually useless.

\begin{figure}
  \centering
  \tiny
  \includesvg[width=\textwidth]{diff30}
  \caption{Scatter plot from comparing effect, cloud coverage and
    optimal effect, using averages over 30 mins}
  \label{fig:scatter30}
\end{figure}

\begin{figure}
  \centering
      \tiny
      \includesvg[width=\textwidth]{diff60}
  \caption{Scatter plot from comparing effect, cloud coverage and
    optimal effect, using averages over 60 mins}
  \label{fig:scatter60}
\end{figure}

To uncover the eventual connection between the percentage of the
effect compared to the amount of clouds all the data points used in
the 30 min scatter plot (the other ratios yield similar results),
the means for the various effects and the standard deviation on the
plus and minus side were mapped.  The plots in figure
\ref{fig:stattotal} seems to indicate extremely vague connections in
the data.  It should be noted that the data at 100\% clouds are
unreliable as there are very few readings there.  Apart from the very
strange outlier at 24\% there could be a vague hint of a connection,
but even if the outlier is just a ``fluke'' in the data, it is still
nothing but a hint.

\begin{figure}
  \centering
  \tiny
  \includesvg[width=\textwidth]{cloudStats}
  \caption{The green line is the mean of the effect/max for the given
    cloud coverage and red are mean $\pm$ standard deviation.  The
    date covers the entire data set.}
  \label{fig:stattotal}
\end{figure}

\begin{figure}
  \centering
  \tiny
  \includesvg[width=\textwidth]{cloudStats06_10}
  \caption{The green line is the mean of the effect/max for the given
    cloud coverage and red are mean $\pm$ standard deviation.  The
    date covers the data from the time ranges from 6.00-10.00}
  \label{fig:stat0610}
\end{figure}

\begin{figure}
  \centering
  \tiny
  \includesvg[width=\textwidth]{cloudStats10_14}
  \caption{The green line is the mean of the effect/max for the given
    cloud coverage and red are mean $\pm$ standard deviation.  The
    date covers the data from the time ranges from 10.00-14.00}
  \label{fig:stat1014}
\end{figure}

\begin{figure}
  \centering
  \tiny
  \includesvg[width=\textwidth]{cloudStats14_18}
  \caption{The green line is the mean of the effect/max for the given
    cloud coverage and red are mean $\pm$ standard deviation.  The
    date covers the data from the time ranges from 14.00-18.00.}
  \label{fig:stat1418}
\end{figure}

During the day the sun will hit the solar panels at different angles.
This may have an effect on the how the panels generate power, the
plots with the mean compared to the standard deviation have also been
compared as seen in figure \ref{fig:stat0610}, \ref{fig:stat1014} and
\ref{fig:stat1418}.  The graphs however still do not indicate a further
connection between the cloud levels and the effect from the PV system.
The standard deviations seem to be lower, but that is most likely due
to the smaller data sets in consideration.

From the given data analysis it seems impossible to draw any
conclusions to a connection between the power generated by a PV system
and cloud coverage.  This result is in contrast to the expected
result.  The data has proven to be way to noisy too draw any
conclusions.  Throughout the analysis there seems to be a slight hint,
so further data analysis may be able to show a connection, by finding
a way to reduce the noise.

%%% Local Variables:
%%% mode: latex
%%% TeX-command-extra-options: "-shell-escape"
%%% TeX-master: "thesis"
%%% End:
